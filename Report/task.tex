\chapter*{Task definition}
The theme for the 3rd semester of Medialogy is Visual Computing. Here subjects about how humans and machines perceive images were taught. In the course \textbf{Image Processing} it was shown how it is possible to process and manipulate digital images in theory and concept, while the course \textbf{Procedural Programming} was about learning the \verb!C++! programming language, as well as the OpenCV framework. To apply the knowledge about image processing in practice, each student were tasked with writing a small program that could process an image in a certain way. It was allowed to use the OpenCV framework to load in images, but the actual image processing algorithm should be written from scratch.

Each mini project was meant as an individual task, and everybody in the group received a different task. The following is the description of the task that I received.

\begin{fancyquotes}Topic \#5: Diagonal Edge Detection

Make a C/C++ program that can find diagonal edges in an image.

Input: Greyscale image
Output: Binary image where the diagonal edges are white (255) and the rest of the pixels black (0)
\end{fancyquotes}