\chapter{Theory about Edge Detection}
\section{Edge definition}
An edge in an image is basically a place in an image where there is a contrast between two points.

\citet{visual_story} describes an edge as the apparent line around the borders of a two-dimensional object.

Another definition is given by \citep{ip_book} who writes that an edge in an image is defined as a position where there is a significant change in gray-level values.

In other words, \textbf{an edge in an image is where the intensity changes dramatically.} A perfect edge would have to be a transition from e.g. black to white over just one pixel, but in the real world this rarely happen, unless it is a binary image where there are only black and white pixels.

\section{The usefulness of edges}
Edges are typically used to define the boundary of an object. This reduces a lot of calculations needed to be done, either by the human brain or a computer,, since it is only necessary to look at the outline and not the whole object. It allows for higher levels of abstraction. This system is used in the way a human perceives the world, using ganglion cell signal changes \citep{perception} \fxnote{bedre skrive + Perception bog}. In machine vision this system is applied, e.g. if a robot needs to recognize and work with a specific object.

\section{The concept of edge detection}
Edges can be described as the slope of blabla bla as shown on pictures \ref{building_gray} and \ref{surface_plot}.

\begin{figure}[htbp]\centering
	\begin{minipage}[b]{0.48\textwidth}\centering
		\includegraphics[width=1.00\textwidth]{C:/Users/Wikzo/Desktop/"Procedural Programming"/Edge_detecting/MyOpenCV/OpenCV_test3/1_grayscale} %Venstre billede
	\end{minipage}\hfill
	\begin{minipage}[b]{0.48\textwidth}\centering
		\includegraphics[width=1.00\textwidth]{C:/Users/Wikzo/Desktop/"Procedural Programming"/Edge_detecting/MyOpenCV/OpenCV_test3/building_surface_plot} %Højre billede
	\end{minipage}\\ %Captions and labels
	\begin{minipage}[t]{0.48\textwidth}
		\caption{The original image seen in grayscale.} %Venstre caption og label
		\label{building_gray}
	\end{minipage}\hfill
	\begin{minipage}[t]{0.48\textwidth}
		\caption{Surface plot point created using ImageJ.} %Højre caption og label
		\label{surface_plot}
	\end{minipage}
\end{figure}

The following is mainly based on \citep{edge_lecture}.
Edge detectors consist of three steps:
\begin{itemize}
\item Noise reduction
\item Edge enhancement
\item Edge localization
\end{itemize}

The first step, \textbf{noise reduction}, can be done using a filter. This could for instance be a median or mean filter. However, there is a dilemma when choosing the size of the filter. A large filter will remove more noise from the image, but it will also remove some of the edges. A smaller filter, on the other hand, keeps more edges but also more noise.

The next step, \textbf{edge enhancement}, calculates the possible candidates for edges.